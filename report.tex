\documentclass{article}
\usepackage{hyperref}
\usepackage[style=apa, backend=biber]{biblatex}
\addbibresource{refs.bib}
\title{A Study of Triangular Minesweeper}
\author{Sean Chua, Lee Hua Jay, Nikola Winata}
\begin{document}

\maketitle

\section{Abstract}

\newpage
\section{Introduction}
Minesweeper is a puzzle video game genre based on logical thinking and guessing that is generally played on personal computers. 
The game features a grid of clickable tiles. Revealing a tile would either show it to 
be a mine, or a safe tile. \parencite{minesweeper} In the case of the latter, the revealed tile would then show the number of tiles it is adjacent to,
diagonally, left and right, or up and down.
In order to win, players must click on all safe tiles without clicking on mines. \\\\
Most variations of the game use a grid of squares to represent this tiles, where the squares are tessellated to form
the overall board. Some examples are the original Minesweeper game, which features the classic grid of squares,
as well as Torus Minesweeper \parencite{torusminesweeper}, a variant of the game in which the grid of squares is mapped onto the surface of a torus,
hence wrapping around, ensuring that the ends of the grid are adjacent to each other. \\\\
Several pieces of mathematical research has been done on the area of Minesweeper. Becerra \parencite{becerra} explored algorithmic approaches
to playing Minesweeper through the implementation of the Single Point approach and the Constraint Satisfaction Problem model for Minesweeper.
He then went on to present two novel implementations of both of the approaches called the double set single point and connected components CSP. 
Becerra then concludes that the coupled subsets CSP model performs the best overall due to its refined probabilistic guessing and 
its ability to find deterministic moves. In 2009, German and Lakshtanov defined a construction of a triangular tiled Minesweeper plane \parencite{laplacian}.
\\\\In this paper, we explore the construction of a Minesweeper plane made of tessellated triangular tiles and its differences in game results as compared to the traditional
Minesweeper board.


\printbibliography
\end{document}